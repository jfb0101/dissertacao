\section{Procedimento}
A eletroestimulação será feita pelo equipamento proposto no músculo tibial anterior da perna dominante. Entre os eletrodos será fixado um sensor acelerômetro para detectar a contração muscular por meio do circuito de biofeedback. Os eletrodos e o acelerômetro serão fixados com fita microporo e será adicionado gel condutor aos eletrodos para melhorar o contato com a pele. Será feita tricotomia na pele.

O sistema será dividido em quatro partes principais: Unidade de Controle, Unidade de Potência, Biofeedback e Interface Gráfica. A unidade de controle será responsável por gerar as formas de onda, em tempo real, por meio dos \ac{dac} de um microcontrolador. A unidade de potência alimentará todo o sistema fornecendo as tensões necessárias para os diversos componentes atuarem. Serão utilizadas fontes de corrente controladas por tensão para gerar os estímulos. O sistema de biofeedback será composto pelo acelerômetro utilizado para detectar a contração muscular. Um algoritmo que lê continuamente os eixos do acelerômetro será escrito para identificar alguma variação considerável em dua posição, detectando assim a contração muscular. A inteface gráfica com o usuário será um aplicativo Android que se comunicará via bluetooth com o eletroestimulador.

O eletroestimulador possuirá 12 canais, permitindo que 12 grupos musculares sejam estimulados simultamente. Para o presente trabalho, somente o músculo tibial anterior será eletroestimulado.

Serão feitos testes de bancada no equipamento, conectando uma carga de 1 $k\Omega$ para simular a resistência da pele. Um osciloscópio será utilizado para verificar se as formas de onda geradas bem como os valores máximos esperados estão sendo gerados pelo sistema.

Serão feitos exames por três avaliadores. Cada avaliador executará os exames de reobase e cronaxia convencionais, observando o músculo até notar a contração muscular, e de forma automática com o equipamento, efetuando teste e re-teste com os mesmos indivíduos com intervalo de dias a definir. O sistema será programado para não interromper automaticamente a eletroestimulação quando o avaliador estiver realizando o exame manual. Logo sem seguida, uma análise de \ac{icc} será conduzida para avaliar a variabilidade das observações entre diferentes observadores e do mesmo observador, comparando com os resultados obtidos pelo equipamento.

Para avaliação da reobase será utilizada uma onda retangular bifásica com largura de pulso de 1s fixa e 2s de descando, amplitude começando em 1 mA, sendo incrementada de 1 em 1 mA.

Para avaliar a cronaxia também será utilizada uma onda retangular bifásica porém com largura de pulso iniciando 20 $\mu$s incrementando de 20 em 20 $\mu$s, com 2s de descanso, amplitude de corrente fixa igual a 2 vezes o valor da reobase.

No presente trabalho não será avaliada a acomodação, pois ela possui o mesmo resultado aparente da reobase, mudando a resposta fisiológica ao estímulo, não alterando assim o resultado para validação do eletrodiagnóstico automático. A remoção da acomodação do estudo também diminuirá o tempo para execução do mesmo, tornando o estudo mais viável.

Os voluntários estarão sentados em uma cadeira com apoio para as costas com a perna dominante esticada sobre outra cadeira.

Os resultados obtidos pelo equipamento serão armazenados em arquivo CSV e enviados para um servidor central para posterior análise estatística.