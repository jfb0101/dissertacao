\section{Introdução}

Em um tecido muscular saudável o nervo motor é despolarizado por estímulos de onda quadrada com duração de 60 a 200 $\mu$s \improvement{citar Recoskie patente}. De forma geral, larguras de pulso de 200 a 500 $\mu$s são utilizadas pois estudos têm demonstrado que esta faixa causa a contração muscular com menor desconforto \improvement{citar De litto A patente}. Tecidos com \ac{den}, como lesão nervosa periférica, lesão central ou pacientes internados em \ac{uti}, necessitam de larguras de pulso superiores a 1000 $\mu$s para causar a mesma contração, devido a alterações neurofisiológicas \improvement{citar sailva pe patente}. Além de uma maior largura de pulso, um aumento na intensidade do limiar de disparo também se torna necessário, uma vez que as mesmas alterações neurofisiológicas dos tecidos comprometem a velocidade de condução dos potenciais de ação através dos axônios \improvement{citar stevens RD}. Logo, torna-se necessário identificar os parâmetros ótimos de eletroesitmulação desses tecidos musculares, por meio de \ac{tede}, pois para um tratamento mais eficaz as áreas afetadas devem ser eletroestimuladas com parâmetros de exceitabilidade mais adequados e, provavelmente, diferentes \improvement{citar peviani SM patente}.

O \ac{tede} é um exame não invasivo utilizado para mensurar a o tempo e intensidade mínimos para causar contração muscular. É uma alternativa à eletromiografia de agulha, um dos exames mais importantes para detectar \ac{pnmdc}, porém raramente realizado por ser invasivo e de alto custo operacional \cite{schuhfried2005}. A eficiência do \ac{tede} se comparado à eletromiografia de agulha, pode chegar de 88 a 100\% \cite{schuhfried2005}. A mensuração da mínima intensidade necessária para causar contração é feita por meio da reobase, onde estímulos elétricos em forma retangular de duração considerada infinita (entre 300 e 1000ms) são aplicados, iniciando com intensidade de 1 mA e aumentando gradativamente até detectar-se a contração. A largura mínima para causar a mesma contração, é mensurada por meio da cronaxia, aplicando estímulos também em forma retangular, com duas vezes o valor da reobase e iniciando em 20 $\mu$s, aumentando até notar-se a contração. Outro parâmetro importante é o índice de acomodação, mensurado pela razão entre os valores da reobase e acomodação. A acomodação tem por objetivo, assim como a reobase, mensurar a intensidade mínima para causar contração muscular, porém com forma de onda exponencial e duração infinita.

Uma vez mensurados os valores ótimos de reobase e cronaxia para um grupo muscular específico, pode-se inclusive realizar terapias intensivas prolongadas que durem de minutos até horas, a fim de evitar a atrofia muscular. Um estudo evidenciou que terapias que utilizem entre 300 e 800 contrações por dia, dividas em um período de 24 horas, podem ser eficientes para tal tratamento \improvement{citar Dow DE patente}.

Os equipamentos existentes no mercado que permitem a realização de \ac{tede} não são totalmente adequados, pois foram concebidos para estimular poucas áreas ao mesmo tempo, com programação prévia dos estímulos, dependência entre os canais, largura de pulso e intensidade de correntes incapazes de gerar contraçoes efetivas em doentes críticos e com impossibilidade de programação mais complexa, por exemplo, para terpias de por períodos prolongados. 

Por exemplo, o Dualpex 071 (manual disponível em $http://quarkmedical.com.br/pdf/DUALPEX_071.pdf$, acessado em 17 de setembro de 2018), pode gerar estímulos elétricos com largura de pulso de até 1 s de duração, adequada para o \ac{tede}, porém sua amplitude máxima é de 69 mA com cargas de no máximo 1 $K\Omega$, possibilitando a eletroestimulação muscular por no máximo 59 minutos, não sendo eficaz também ao gerar a forma de onda exponencial necessária para mensuração da acomodação.

Outro dipositivo que não adequa totalmente às necessidades do \ac{tede} para doentes críticos, é equipamento descrito na patente estadunidense US6324432, voltado para aplicações esportivas, sobretudo treinamento passivo e reeducação de tecidos musculares atrofiados. Este equipamento possui sensores do tipo eletromiógrafo e acelerômetro, para medição das reações musculares dos tecidos musculares estimulados. No entanto, não possibilita a medição da atividade neuromuscular com um controle em malha fechada, bem como a atualização dos estímulos elétricos. Possui no máximo 2 canais que não permitem a programação da eletroesitmulação por períodos prolongados, algo necessário em terapias intensivas como citado anteriormente.

Diversas outras patentes (US8565888,US8285381B2,US20160051817A1) propõem eletroestimuladores mas cada um deles possuem algum fator que o torna ineficiente, como número limitado de canais, impossibilidade da programação da eletroestimulação, impossibilidade de gerar estímulos com largura de pulso e/ou intensidades adequadas, dentre outros.

Uma patente do equpamento aqui proposto, desenvolvido pelo mesmo grupo de pesquisa, já foi depositada no \ac{inpi}, sob o número BR 10 2017 026510 2 está sob análise.

Além da falta de equipamentos, outro fator que compromete o uso de \ac{tede} é sua aplicação na prática clínica e em \ac{uti}. Como é necessário validar visualmente a contração muscular, podem haver diferentes interpretações, podendo causar certa variabilidade entre as observações dos valores de largura de pulso e intensidade mensurados para o mesmo grupo muscular pelo mesmo observador ou entre diferentes observadores. O tempo em \ac{uti} também é um fator importante, e o eletrodiagnóstico convencional pode levar muito tempo para ser feito. O equipamento aqui descrito propõe uma solução para este problema ao utilizar um sistema de malha fechada com biofeedback com acelerômetro fixado ao músculo sendo avaliado, que detecta automaticamente a contração muscular, por meio da medição do desvio padrão anormal causado pela contração no momento da estimulação. A sensibilidade do acelerômetro permite detectar a contração tão logo ela aconteça, além de possibilitar resultados mais precisos e com menor variabilidade entre repetições do mesmo exame para o mesmo grupo muscular, além de tornar o eletrodiagnóstico mais rápido e factível nos ambientes já citados. 

A hipótese deste estudo é que os valores de reobase e cronaxia obtidos pelo eletrodiagnóstico automático sejam tão bons quanto ou melhores, do ponto de vista quantitativo e qualitativo, que o eletrodiagnóstico convencional.

Para validar a eficiência e segurança do equpamento proposto, testes com indivíduos saudáveis serão conduzidos, excluindo do estudo pacientes com qualquer tipo de \ac{den}. Apesar do equipamento ter sido desenvolvido seguindo as normas preconizadas pela Anvisa faz-se necessário testes de segurança.

O objetivo deste estudo é desenvolver um eletroestimulado adequado para obtenção dos valores adequados de reobase, cronaxia e acomodação para uso no \ac{tede} com biofeedback, tornando o exame mais rápido, menos operador-dependente, por conseguinte mais confiável e mais simples de ser utilizado na prática clínica



