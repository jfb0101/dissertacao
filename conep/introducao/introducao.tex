\section{Introdução}

O \ac{tede} é um exame não invasivo de baixo custo usado para mensurar a reobase, cronaxia e acomodação \improvement{citar referência}. A reobase é a menor intensidade de corrente para produzir contração muscular mínima, sendo o estímulo uma onda quadrada de duração infinita (entre 300 e 1000ms) e período de descanso igual a 2000ms, podendo ser bifásica ou não. A cronaxia é a menor duração de pulso, também com uma onda quadrada bifásica ou não necessária para produzir contração muscular mínima, sendo sua intensidade igual ao dobro da reobase com intervalos de 2000ms. Já acomodação, assim como a reobase, é a menor intensidade de corrente que produz contração muscular mínima, porém com forma de onda exponencial, largura infinita e período de descanso também de 2000ms \improvement{citar referência}.

Pode ser utilizado para diagnosticar \ac{den}  \cite{schuhfried2005}. O \ac{tede} é uma alternativa ao já conhecido e importante exame de eletromiografia com eletrodo de agulha, sendo este um exame mais caro, que exige equipe técnica especializada e é invasivo. Em relação à eletromiografia com eletrodo de agulha a sensibilidade do \ac{tede} varia entre 88 e 100\% \cite{schuhfried2005}.

Geralmente é utilizado como ferramenta de diagnóstico, sendo valores de cronaxia superiores a 1000us caracterizam a presença de \ac{den} \cite{sluga2002}, mas também pode ser utilizado para fornecer parâmetros para terapias intensivas por meio de \ac{eenm}. Terapias utilizando \ac{eenm} com base na cronaxia podem melhorar a eficiência do tratamento \cite{silvaPE}

O problema do \ac{tede} está na sua aplicação prática e na sua reprodutibilidade. Ele é operador-dependente, sendo necessária a inspeção visual da região muscular sendo eletroestimulada. Esta inspeção depende de um profissional especializado neste tipo de exame e pode sofrer variações de profissional para profissional, devido à diferentes interpretações. O tempo levado para efetuar o exame também é um fator que compromete sua aplicação.

Uma possível solução para tornar o exame mais factível seria a automatização dos testes de reobase e cronaxia por meio de um equipamento especializado para tal.

