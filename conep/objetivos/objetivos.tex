\section{Objetivos}

Objetivo geral:

Desenvolver um equipamento capaz de automatizar a execução dos exames de reobase, cronaxia e acomodação por meio de um sistema de biofeedback, que identifica o momento exato no qual a contração muscular ocorreu e registra o respectivo valor de intensidade ou largura de pulso, diminuindo assim a variabilidade intra e entre observadores e tornando o exame mais prático, rápido e eficiente.

Objetivos específicos:

\begin{itemize}
    \item Desenvolver um algoritmo de detecção de contração muscular que identifique tal contração tão logo ela aconteça utilizando um sensor acelerômetro.
    \item Identificar a região de maior excitabilidade muscular para fixação dos eletrodos e do sistema de biofeedback.
    \item Desenvolver um equipamento com 12 canais independentes para estimular simultaneamente diferentes grupos musculares.
    \item Desenvolver uma interface gráfica simples e intuitiva para rápida adequação dos utilizadores do sistema.
    \item Desenvolver e armazenar os resultados obtidos em um sistema disponível via internet para um rápido e centralizado acesso aos dados quando necessário
    \item Analisar quantitativamente e qualitativamente por meio do teste de \ac{icc} se os valores obtidos pelo equipamento são tão assertivos ou superioes aos valores obtidos pela exame convencional.
    \item Validar as formas de onda geradas em testes de bancada utilizando osciloscópio e cargas que simulem a resistência da pele.
    \item Avaliar a usabilidade do sistema como um todo por profissionais da área de saúde e fisioteriapia que farão uso do mesmo em ambientes clínicos
\end{itemize}