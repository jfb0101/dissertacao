\section{Introdução}

O \ac{tede} é um exame não invasivo que detecta alterações de excitabilidade neuromuscular quantificando as respostas evocadas pelo nervo e pelo músculo com parâmetros de corrente elétrica, sendo composto pela mensuração da reobase, cronaxia e índice de acomodação \cite{sluga2002,schuhfired2005,lee2013}.

Em pacientes com \ac{lm}, a reabilitação requer o diagnóstico do nível e da gravidade da lesão por meio da identificação das zonas sensório-motoras íntegras, o que permite o tratamento e o prognóstico, bem como acompanhar a recuperação do sistema nervoso. Apesar da escala da \ac{asia} ser o método padronizado para essa avaliação \cite{van2011}, sua aplicação é limitada a pacientes não-cooperativos e ela não fornece todos os dados necessários para o entendimento de alterações fisiológicas e estruturais secundárias. Sendo assim, para este tipo de tratamento, os exames eletrofisiológicos que avaliam a condução nervosa são muito utilizados na busca por comprometimentos neuromusculares.

Pacientes que estejam há muito tempo sem mobilidade devido à, por exemplo, internação em \ac{cti} podem apresentar perda de massa muscular e atrofia.

Em todos os casos citados, terapias envolvendo \ac{eenm} são de grande valia para ajudar na avaliação, prevenção, diagnóstico ou melhoria na condição neuromuscular.

A eficiência do \ac{eenm} pode ser avaliada pelo torque causado devido à eletroestimulação artificial aplicada, sendo este torque diretamente relacionado à largura de pulso e amplitude dos estímulos.

Porém não existe concenso sobre os valores ideiais para tais parâmetros, sendo obtidos com base nos testes de reobase, cronaxia e índice de acomodação. 

A reobase é a menor intensidade de corrente com forma de pulso retangular e largura de pulso infinita para provocar contração muscular visível. A cronaxia é o menor tempo necessário para causar também contração muscular visível, com amplitude de corrente quadrada igual a duas vezes a reobase. A acomodação é a menor intensidade de corrente necessária para produzir contração muscular, mas com a forma de onda exponencial e largura de infinita, avaliando assim a capacidade do músculo em responder a altas intensidades e pulsos de crescimento lento. O índice de acomodação é a relação entre a acomodação e a reobase \cite{sluga2002}.

Apesar desses testes serem de grande importância, sua aplicabilidade clínica pode ser comprometida por serem operador-dependente e necessitarem de inspeção visual, podendo gerar falsos positivos ou falsos negativos, conferindo grande variabilidade aos testes.

Logo, ter um equipamento que possa tornar esses exames mais eficientes, rápidos e menos operador-dependente torna-se importante na aplicação clínica e acadêmica.