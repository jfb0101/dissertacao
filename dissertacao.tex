\documentclass[a4paper,10pt]{article}
%\documentclass[a4paper,10pt]{scrartcl}

\usepackage[utf8]{inputenc}
\usepackage{acro}

\usepackage[colorinlistoftodos,prependcaption,textsize=tiny]{todonotes}
\newcommand{\unsure}[1]{\todo[linecolor=red,backgroundcolor=red!25,bordercolor=red]{#1}}
\newcommand{\change}[1]{\todo[linecolor=blue,backgroundcolor=blue!25,bordercolor=blue]{#1}}
\newcommand{\info}[1]{\todo[linecolor=green,backgroundcolor=green!25,bordercolor=green]{#1}}
\newcommand{\improvement}[1]{\todo[linecolor=Plum,backgroundcolor=Plum!25,bordercolor=Plum]{#1}}
\newcommand{\thiswillnotshow}[1]{\todo[disable]{#1}}

\title{}
\author{}
\date{}

\pdfinfo{%
  /Title    ()
  /Author   ()
  /Creator  ()
  /Producer ()
  /Subject  ()
  /Keywords ()
}

\DeclareAcronym{tede} {
	short = TEDE,
	long = Teste de Eletrodiagnóstico de Estímulo,
	class = abbrev
}

\DeclareAcronym{cti} {
	short = CTI,
	long = Centro de Tratamento Intensivo,
	class = abbrev
}

\DeclareAcronym{eenm} {
	short = NMES,
	long =  Estimulação Elétrica Neuromuscular,
	class = abbrev
}

\DeclareAcronym{lm} {
	short = LM,
	long =  Lesão Medular,
	class = abbrev
}

\DeclareAcronym{asia} {
	short = ASIA,
	long =  American Spinal Injury Association,
	class = abbrev
}

\DeclareAcronym{icc} {
	short = ICC,
	long =  Intraclass Correlation Coefficient - Coeficiente de Correlação Intraclasse,
	class = abbrev
}

\DeclareAcronym{dac} {
	short = DAC,
	long =  Digital Analog Converter - Conversor Analógico Digital,
	class = abbrev
}

\begin{document}
\maketitle



\section{Introdução}

O \ac{tede} é um exame não invasivo que detecta alterações de excitabilidade neuromuscular quantificando as respostas evocadas pelo nervo e pelo músculo com parâmetros de corrente elétrica, sendo composto pela mensuração da reobase, cronaxia e índice de acomodação \cite{sluga2002,schuhfired2005,lee2013}.

Estes testes possuem diversas aplicações para diagnósticos de neuropatias...

Em pacientes com \ac{lm}, a reabilitação requer o diagnóstico do nível e da gravidade da lesão por meio da identificação das zonas sensório-motoras íntegras, o que permite o tratamento e o prognóstico, bem como acompanhar a recuperação do sistema nervoso. Apesar da escala da \ac{asia} ser o método padronizado para essa avaliação \cite{van2011}, sua aplicação é limitada a pacientes não-cooperativos e ela não fornece todos os dados necessários para o entendimento de alterações fisiológicas e estruturais secundárias. Sendo assim, para este tipo de tratamento, os exames eletrofisiológicos que avaliam a condução nervosa são muito utilizados na busca por comprometimentos neuromusculares.

Pacientes com neuropatia diabética...

Pacientes que estejam há muito tempo sem mobilidade devido à, por exemplo, internação em \ac{cti} podem apresentar perda de massa muscular e atrofia.

Em todos os casos citados, terapias envolvendo \ac{eenm} são de grande valia para ajudar na avaliação, prevenção, diagnóstico ou melhoria na condição neuromuscular.

A eficiência do \ac{eenm} pode ser avaliada pelo torque causado devido à eletroestimulação artificial aplicada, sendo este torque diretamente relacionado à largura de pulso e amplitude dos estímulos.

Porém não existe concenso sobre os valores ideiais para tais parâmetros, sendo obtidos com base nos testes de reobase, cronaxia e índice de acomodação. Tais testes são operador-dependente, necessitam de inspeção visual, podendo gerar falsos positivos ou falsos negativos, conferindo grande variabilidade aos testes.

A reobase é a menor intensidade de corrente com forma de pulso retangular e largura de pulso infinita para provocar contração muscular visível. A cronaxia é o menor tempo necessário para causar também contração muscular visível, com amplitude de corrente quadrada igual a duas vees a reobase. A acomodação é a menor intensidade de corrente necessária para produzir contração muscular, mas com a forma de onda exponencial e largura de infinita, avaliando assim a capacidade do músculo em responder a altas intensidades e pulsos de crescimento lento. O índice de acomodação é a relação entre a acomodação e a reobase \cite{sluga2002}.

Visto a importância do \ac{tede}, ter um equipamento que possa torná-lo mais eficiente e menos operador-dependente torna-se importante na aplicação clínica e acadêmica.

\section{Objetivos}

\subsection{Objetivo Geral}
Desenvolver e validar um eletroestimulador multicanais automático com biofeedback para  substituir o eletrodiagnóstico manual feito atualmente com a mesma ou maior eficiência.
\subsection{Objetivos específicos}
\begin{itemize}
  \item Diminuir o tempo de execução dos exames.
  \item Aumentar a reprodutibilidade dos exames por meio da avaliação intra e inter observadores.’
  \item Identificar mais facilmente os pontos de maior excitabilidade muscular.
\end{itemize}

\section{Variáveis do estudo}
Serão obtidos idade e sexo dos voluntários.

\section{Critérios de inclusão}

Serão aplicados os seguintes critérios de inclusão:

\begin{itemize}
  \item Voluntários com mais de 18 anos
  \item Homens e mulheres  
\end{itemize}

\section{Critérios de exclusão}

Serão aplicados os seguintes critérios de exclusão:

\begin{itemize}
  \item Voluntários que tiverem feito atividade física que comprometa o músculo tibial anterior
  \item Voluntários com algum tipo de deficiência física e/ou \unsure{quero falar de deficiências que afetem o sistema nervoso periférico. como falar?}

\end{itemize}


\section{Procedimento}
A eletroestimulação será feita pelo equipamento proposto no músculo tibial anterior da perna dominante. Entre os eletrodos será fixado um sensor acelerômetro para detectar a contração muscular. Os eletrodos e o acelerômetro serão fixados com fita microporo e será adicionado gel condutor aos eletrodos para melhor contato com a pele. Será feita \unsure{esqueci a palavra bonita para “raspar a perna”} no músculo.

Os exames serão feitos simultaneamente pelo avaliador e pelo sistema, afim de comparar posteriormente o momento da detecção da contração muscular entre ambos. Uma análise intra e inter obervadores será conduzida.

Em todos os 3 exames a eletroestimulação é interrompida quando ambos, o avaliador e o sistema, detectarem a contração muscular.

Para avaliação da reobase será utilizada uma onda retangular bifásica com largura de pulso de 1s fixa e 2s de descando, amplitude começando em 1 mA, sendo incrementada de 1 em 1 mA.

Para avaliar a cronaxia também será utilizada uma onda retangular bifásica porém com largura de pulso iniciando 20 $\mu$s incrementando de 20 em 20 $\mu$s, com 2 s de descanso, amplitude de corrente fixa igual a 2 vezes o valor da reobase.

Na acomodação a onda utilizada será exponencial bifásica com período de 1s e 2s de descanso, amplitude iniciando em 1 mA aumentando de 1 em 1 mA.
O índice de acomodação será determinado pela razão entre a acomodação e a reobase.

Os voluntários estarão sentados em uma cadeira com apoio para as costas com a perna dominante esticada sobre outra cadeira.


\bibliographystyle{unsrt}
\bibliography{bibliografia}

\end{document}
