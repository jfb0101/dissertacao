\documentclass[a4paper,10pt]{article}
%\documentclass[a4paper,10pt]{scrartcl}

\usepackage[utf8]{inputenc}

\usepackage[colorinlistoftodos,prependcaption,textsize=tiny]{todonotes}
\newcommand{\unsure}[1]{\todo[linecolor=red,backgroundcolor=red!25,bordercolor=red]{#1}}
\newcommand{\change}[1]{\todo[linecolor=blue,backgroundcolor=blue!25,bordercolor=blue]{#1}}
\newcommand{\info}[1]{\todo[linecolor=green,backgroundcolor=green!25,bordercolor=green]{#1}}
\newcommand{\improvement}[1]{\todo[linecolor=Plum,backgroundcolor=Plum!25,bordercolor=Plum]{#1}}
\newcommand{\thiswillnotshow}[1]{\todo[disable]{#1}}

\title{}
\author{}
\date{}

\pdfinfo{%
  /Title    ()
  /Author   ()
  /Creator  ()
  /Producer ()
  /Subject  ()
  /Keywords ()
}

\begin{document}
\maketitle

\section{Objetivos}

\subsection{Objetivo Geral}
Validar se o eletrodiagnóstico automático pode substituir o eletrodiagnóstico manual feito atualmente com a mesma ou maior eficiência.
\subsection{Objetivos específicos}
\begin{itemize}
  \item Diminuir o tempo de execução dos exames.
  \item Aumentar a reprodutibilidade dos exames por meio da avaliação intra e inter observadores.’
  \item Identificar mais facilmente os pontos de maior excitabilidade muscular.
\end{itemize}

\section{Variáveis do estudo}
Serão obtidos idade e sexo dos voluntários.

\section{Critérios de exclusão}

\section{Procedimento}
A eletroestimulação será feita pelo equipamento proposto no músculo tibial anterior da perna dominante. Entre os eletrodos será fixado um sensor acelerômetro para detectar a contração muscular. Os eletrodos e o acelerômetro serão fixados com fita microporo e será adicionado gel condutor aos eletrodos para melhor contato com a pele. Será feita \unsure{esqueci a palavra bonita para “raspar a perna”} no músculo.

Os exames serão feitos simultaneamente pelo avaliador e pelo sistema, afim de comparar posteriormente o momento da detecção da contração muscular entre ambos. Uma análise intra e inter obervadores será conduzida.

Em todos os 3 exames a eletroestimulação é interrompida quando ambos, o avaliador e o sistema, detectarem a contração muscular.

Para avaliação da reobase será utilizada uma onda retangular bifásica com largura de pulso de 1s fixa e 2s de descando, amplitude começando em 1 mA, sendo incrementada de 1 em 1 mA.

Para avaliar a cronaxia também será utilizada uma onda retangular bifásica porém com largura de pulso iniciando 20 $\mu$s incrementando de 20 em 20 $\mu$s, com 2 s de descanso, amplitude de corrente fixa igual a 2 vezes o valor da reobase.

Na acomodação a onda utilizada será exponencial bifásica com período de 1s e 2s de descanso, amplitude iniciando em 1 mA aumentando de 1 em 1 mA.
O índice de acomodação será determinado pela razão entre a acomodação e a reobase.

Os voluntários estarão sentados em uma cadeira com apoio para as costas com a perna dominante esticada sobre outra cadeira.



\end{document}
