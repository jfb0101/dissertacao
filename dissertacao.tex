\documentclass[a4paper,10pt]{article}
%\documentclass[a4paper,10pt]{scrartcl}

\usepackage[utf8]{inputenc}

\title{}
\author{}
\date{}

\pdfinfo{%
  /Title    ()
  /Author   ()
  /Creator  ()
  /Producer ()
  /Subject  ()
  /Keywords ()
}

\begin{document}
\maketitle

\section{Objetivos}

\subsection{Objetivo Geral}
Validar se o eletrodiagnóstico automático pode substituir o eletrodiagnóstico manual feito atualmente com a mesma ou maior eficiência.
\subsection{Objetivos específicos}
Diminuir o tempo de execução dos exames.
Aumentar a reprodutibilidade dos exames por meio da avaliação intra e inter observadores.’
Identificar mais facilmente os pontos de maior excitabilidade muscular.

\section{Variáveis do estudo}
Serão obtidos idade e sexo dos voluntários

\section{Procedimento}
A eletroestimulação será feita pelo equipamento proposto no músculo tibial anterior. Entre os eletrodos será fixado um sensor acelerômetro para detectar a contração muscular. Os estímulos aplicados serão ondas retangulares bifásicas com amplitude entre 1 mA e 90 mA e largura de pulso entre 20 $\mu$s e 1 s. Os eletrodos serão fixados com fita microporo e será adicionado gel condutor para melhor contato com a pele. Será feita blablabla  no músculo.
Em todos os 3 exames a eletroestimulação é interrompida quando detecta-se uma contração muscular.
Para avaliação da reobase será utilizada uma onda retangular bifásica com largura de pulso de 1 s fixa e 2 s de descando, amplitude começando em 1 mA, sendo incrementada de 1 em 1 mA.
Para avaliar a cronaxia também será utilizada uma onda retangular bifásica porém com largura de pulso iniciando 20 $\mu$s incrementando de 20 em 20 $\mu$s, com 2 s de descanso, amplitude de corrente fixa igual a 2 vezes o valor da reobase.
Na acomodação a onda utilizada será exponencial bifásica com período de 1 s e 2 s de descanso, amplitude iniciando em 1 mA aumentando de 1 em 1 mA.


\end{document}
