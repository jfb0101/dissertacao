\section{Procedimento}
A eletroestimulação será feita pelo equipamento proposto no músculo tibial anterior da perna dominante. Entre os eletrodos será fixado um sensor acelerômetro para detectar a contração muscular por meio do circuito de biofeedback. Os eletrodos e o acelerômetro serão fixados com fita microporo e será adicionado gel condutor aos eletrodos para melhorar o contato com a pele. Será feita \unsure{esqueci a palavra bonita para “raspar a perna”} na pele.

Serão feitos exames por três avaliadores. Cada avaliador executará os exames de reobase e cronaxia convencionais, observando o músculo até notar a contração muscular, e de forma automática com o equipamento, efetuando teste e re-teste com os mesmos indivíduos com intervalo de dias a definir. O sistema será programado para não interromper automaticamente a eletroestimulação quando o avaliador estiver realizando o exame manual. Logo sem seguida, uma análise de \ac{icc} será conduzida para avaliar a variabilidade das observações entre diferentes observadores e do mesmo observador, comparando com os resultados obtidos pelo equipamento.

Para avaliação da reobase será utilizada uma onda retangular bifásica com largura de pulso de 1s fixa e 2s de descando, amplitude começando em 1 mA, sendo incrementada de 1 em 1 mA.

Para avaliar a cronaxia também será utilizada uma onda retangular bifásica porém com largura de pulso iniciando 20 $\mu$s incrementando de 20 em 20 $\mu$s, com 2s de descanso, amplitude de corrente fixa igual a 2 vezes o valor da reobase.

No presente trabalho não será avaliada a acomodação, pois ela possui o mesmo resultado aparente da reobase, mudando a resposta fisiológica ao estímulo, não alterando assim o resultado para validação do eletrodiagnóstico automático. A remoção da acomodação do estudo também diminuirá o tempo para execução do mesmo, tornando o estudo mais viável.

Os voluntários estarão sentados em uma cadeira com apoio para as costas com a perna dominante esticada sobre outra cadeira.

Os resultados obtidos pelo equipamento serão armazenados em arquivo CSV e enviados para um servidor central para posterior análise estatística.